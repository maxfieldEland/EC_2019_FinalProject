\section{Introduction}

Each year, between 4 - 8 million acres of land are damaged by wildfires. Just in the past 10 years, this has represented a \$5.1 billion cost in 
infrastructural damage repair \cite{verisk}. 
Forest fires are a complex natural disaster that involve combustion of natural fuels in a wild land subject 
to complex meteorological conditions. Building predictive models to aid in wildfire preparation and containment efforts
is increasingly important. The coupling of the atmospheric systems and the spreading
of fire makes modeling these systems non trivial.  Current cutting edge wild fire models are complex derivations of fundamental 
physical laws \cite{FIRETEC},\cite{Sero-GuillaumeO},\cite{mandel2011coupled}. To reduce computation time, increase accuracy and leverage the advances in satellite
imagery, recent work has been done to model wildfire dynamics with machine learning or evolutionary strategies. This area has seen some great success with 
increased accuracy of perimeter prediction from historic fires. 


While satellite data does exist, the time and space resolution is highly variant in image sets documenting a given fire event. This makes validating fire spreading models
very difficult. We propose an agent based model with an evolving spreading function that predicts the behavior of simulated fire perimeter over time. Then, 
we test the method by simulating the 2016 Fort McMurray Wildfire that occurred during the months of May - September in Fort McMurray, Alberta. 

Agent Based models have been used to describe macro behaviors by prescribing micro behaviors to independent landscapes making decisions in discrete time.
By evolving the function that governs agent behavior, agent based models can be used to predict macro behavior based on ground truth data. \cite{zhong2014automatic}
et all modeled crowd dynamics by evolving the agent rule set through a symbolic regression. Additionally, geometric semantic genetic programming has been used to estimate
the total area burned for a given fire event \cite{castelli2015Mauro}.  



\section{Methods}
\subsection{The Genetic Program}

\subsection{Synthetic Fire Prediction}
\subsection {2016 Fort McMurray Wildfire}

\section{Results}


\section{Discussion}



\section{Conclusions}
This paragraph will end the body of this sample document.
Remember that you might still have Acknowledgments or
Appendices; brief samples of these
follow.  There is still the Bibliography to deal with; and
we will make a disclaimer about that here: with the exception
of the reference to the \LaTeX\ book, the citations in
this paper are to articles which have nothing to
do with the present subject and are used as
examples only.
%\end{document}  % This is where a 'short' article might terminate



\appendix
%Appendix A
\section{Headings in Appendices}
The rules about hierarchical headings discussed above for
the body of the article are different in the appendices.
In the \textbf{appendix} environment, the command
\textbf{section} is used to
indicate the start of each Appendix, with alphabetic order
designation (i.e., the first is A, the second B, etc.) and
a title (if you include one).  So, if you need
hierarchical structure
\textit{within} an Appendix, start with \textbf{subsection} as the
highest level. Here is an outline of the body of this
document in Appendix-appropriate form:
\subsection{Introduction}
\subsection{The Body of the Paper}
\subsubsection{Type Changes and  Special Characters}
\subsubsection{Math Equations}
\paragraph{Inline (In-text) Equations}
\paragraph{Display Equations}
\subsubsection{Citations}
\subsubsection{Tables}
\subsubsection{Figures}
\subsubsection{Theorem-like Constructs}
\subsubsection*{A Caveat for the \TeX\ Expert}
\subsection{Conclusions}
\subsection{References}
Generated by bibtex from your \texttt{.bib} file.  Run latex,
then bibtex, then latex twice (to resolve references)
to create the \texttt{.bbl} file.  Insert that \texttt{.bbl}
file into the \texttt{.tex} source file and comment out
the command \texttt{{\char'134}thebibliography}.
% This next section command marks the start of
% Appendix B, and does not continue the present hierarchy
\section{More Help for the Hardy}

Of course, reading the source code is always useful.  The file
\path{acmart.pdf} contains both the user guide and the commented
code.

\begin{acks}
  The authors would like to thank Dr. Yuhua Li for providing the
  MATLAB code of the \textit{BEPS} method.

  The authors would also like to thank the anonymous referees for
  their valuable comments and helpful suggestions. The work is
  supported by the \grantsponsor{GS501100001809}{National Natural
    Science Foundation of
    China}{http://dx.doi.org/10.13039/501100001809} under Grant
  No.:~\grantnum{GS501100001809}{61273304}
  and~\grantnum[http://www.nnsf.cn/youngscientists]{GS501100001809}{Young
    Scientists' Support Program}.

\end{acks}
